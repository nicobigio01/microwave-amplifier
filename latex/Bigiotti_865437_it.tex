\documentclass[12pt]{article}
\usepackage[italian]{babel}
\usepackage{amsmath}
\usepackage{siunitx}
\usepackage{derivative}
\usepackage{graphicx}
\usepackage[a4paper, total={6.5in, 10in}]{geometry}
\usepackage{caption}
\usepackage{subcaption}
\usepackage{float}

\title{Sviluppo e caratterizzazione di un amplificatore RF per rivelatori criogenici}
\author{Nicolas Bigiotti\\{\small Matr. 865437}\\{\small Cell. 3452438264 }\\[0.4cm]{\small Relatore: Claudio Gotti}\\{\small Correlatore: Gianluigi Ezio Pessina, Paolo Carniti}}

\date{19-20 Luglio 2023 - Corso di Laurea triennale in Fisica}


\begin{document}
\maketitle

Questo elaborato si pone come obiettivo la realizzazione di un amplificatore in radiofrequenza (RF) con transistor bipolari a eterogiunzione SiGe e la successiva caratterizzazione a bassa temperatura. Amplificatori di questo tipo sono richiesti in molti ambiti sia di ricerca che industriali.

Lo studio dei dispositivi a microonde è reso complesso dalle piccole lunghezze d'onda dei segnali coinvolti, comparabili con le dimensioni fisiche del circuito. Prevedere e studiare il comportamento a temperature molto inferiori a quelle di normale esercizio è complicato dalla mancanza di documentazione ufficiale sui parametri del transistor.

Un tipico amplificatore RF è composto da una rete di ingresso con il compito di adattare l'impedenza del generatore a quella in ingresso del transistor, una rete di uscita per adattare l'impedenza d'uscita del transistor all'impedenza di carico e infine il transistor stesso, che fornisce l'amplificazione. A differenza di quanto accade con dispositivi utilizzati in applicazioni di bassa frequenza, per studiare la risposta di piccolo segnale non viene utilizzato un modello a componenti concentrati, ma è più comodo considerare i parametri di scattering che descrivono trasmissione e riflessione dei segnali elettrici all'ingresso e uscita. Per una tipica configurazione a emettitore comune, come quella usata in questo lavoro tali parametri sono forniti dal costruttore del transistor e sono sfruttati in simulazione per dimensionare le reti di ingresso e uscita.

Il primo passaggio di progettazione consiste nel valutare la stabilità della configurazione. La potenziale instabilità è dovuta alla presenza di una reazione interna al transistor che dipende dalla frequenza e dal punto di lavoro e ad eventuali riflessioni introdotte dalle reti di adattamento. Per il transistor utilizzato selezionando una banda stretta centrata a $\SI{5}{\giga\hertz}$ e polarizzando con una corrente di collettore $I_{C}=\SI{8}{\milli\ampere}$ e una tensione di collettore-emettitore $V_{CE}=\SI{3.0}{\volt}$ è verificata l'ipotesi di stabilità incondizionata, vale a dire che indipendentemente dalle riflessioni introdotte dalle reti di adattamento l'amplificatore risulterà stabile e si ha piena libertà di progettazione di quest'ultime.

Garantita la stabilità si procede progettando le reti di adattamento in modo da ottenere il massimo trasferimento di potenza tra ingresso e uscita alla frequenza di esercizio. Per la realizzazione è stata utilizzata la tecnica a stub singolo che sfrutta segmenti di linea di trasmissione proporzionali in lunghezza a frazioni di lunghezza d'onda del segnale di interesse per ottenere l'adattamento di impedenza voluto. La verifica della stabilità del transitor e il dimensionamento delle reti di adattamento sono stati resi possibili dalla scrittura di alcuni script MATLAB, realizzati grazie ad un toolbox specifico per la progettazione RF.
Una volta definito lo schema elettrico è stata realizzata la scheda con il software open source KiCad in modo da rendere possibile la produzione del circuito stampato da una azienda specializzata. Una volta ricevuta la scheda, i componenti precedentemente selezionati sono stati montati su di essa.

Con l'ausilio di un analizzatore vettoriale di rete è stata fatta una misura completa dei parametri di scattering del circuito realizzato a temperatura ambiente, osservando un buon accordo con i valori previsti in fase di progetto. Alla frequenza di progetto di $\SI{5}{\giga\hertz}$ le impedenze di ingresso e uscita sono risultate molto vicine a $\SI{50}{\ohm}$, con un guadagno pari a $\SI{12}{\decibel}$. Inoltre è stato possibile verificare la funzionalità dell'amplificatore osservando nel dominio del tempo il segnale di uscita in presenza di una sinusoide in ingresso. Le misure sono state ripetute a temperatura criogenica fino a $\SI{4}{\kelvin}$. L'amplificatore è rimasto funzionante  durante tutto il graduale processo di raffreddamento. Le misure dei parametri di scattering criogenici hanno messo in evidenza l'efficacia delle reti di adattamento anche a bassa temperatura, mentre si è osservata una diminuzione del guadagno del transistor, che si è attestato a circa $\SI{7}{\decibel}$ a $\SI{4}{\kelvin}$.

Sulla stessa scheda è stato sviluppato un secondo amplificatore non più basato sull'uso degli stub. Si è mantenuto lo stesso punto di lavoro precedente per il transistor pari a $I_{C}=\SI{8}{\milli\ampere}$ e $V_{CE}=\SI{3.0}{\volt}$ ma si è deciso di centrare la rete di amplificazione ad una frequenza di lavoro di $\SI{1.6}{\giga\hertz}$. A questa frequenza il transistor potrebbe risultare potenzialmente instabile ed è stato necessario porre maggiore attenzione alla progettazione delle reti di adattamento di impedenza, basate questa volta sull'adozione di componenti discreti e non più stub, in modo da permettere maggiore flessibilità nella sagomatura delle reti di adattamento. Le prestazioni dell'amplificatore sono state misurate a temperatura ambiente e in azoto liquido ottenendo un guadagno di circa $\SI{19}{\decibel}$ in entrambe le condizioni.

In conclusione tre immagini che meglio riassumono il lavoro svolto: (a) la scheda nel suo involucro di rame, (b) e (c) i risultati delle misure a freddo.
\begin{figure}[H]
    \centering
    \begin{subfigure}[t]{0.3\textwidth}
        \centering
        \includegraphics[width=\textwidth]{img/IMG_2169.jpg}
        \caption{La scheda completa dopo la saldatura e il montaggio nell'involucro metallico necessario per un buon contatto termico con il criostato.}
    \end{subfigure}
    \hfill
    \begin{subfigure}[t]{0.33\textwidth}
        \centering
        \includegraphics[width=\textwidth]{img/meas1.png}
        \caption{Parametri di scattering $S_{11}$, $S_{22}$ e $S_{21}$ misurati a $\approx\SI{4}{\kelvin}$ dell'amplificatore con stub, al guadagno in figura va aggiunta l'attenuazione dei cavi coassiali che risulta oscillare tra i $\SI{5}{\decibel}$ e gli $\SI{8}{\decibel}$.}
    \end{subfigure}
    \hfill
    \begin{subfigure}[t]{0.33\textwidth}
        \centering
        \includegraphics[width=\textwidth]{img/LNA_L_77K.png}
        \caption{Parametri di scattering $S_{11}$, $S_{22}$, $S_{12}$ e $S_{21}$ misurati a $\SI{77}{\kelvin}$ dell'amplificatore con componenti discreti.}
    \end{subfigure}
\end{figure}

\end{document}