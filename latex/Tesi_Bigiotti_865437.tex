\documentclass[12pt]{article}
\usepackage[italian]{babel}
\usepackage{amsmath}
\usepackage{siunitx}
\usepackage{derivative}
\usepackage{graphicx}
\usepackage[a4paper, total={6in, 9in}]{geometry}
\usepackage{caption}
\usepackage{subcaption}
\usepackage{float}
\usepackage{derivative}

\begin{document}
\section{Introduzione}
\subsection{Motivazione e Contesto}
Negli ultimi decenni si è visto un notevole sviluppo della tecnologia legata ai sistemi a microonde dovuta a crescenti necessità sia industriali che di ricerca.
Quantum computing, radioastronomia e comunicazioni satellitari sono alcuni dei campi più importanti dove dispositivi a microonde devono lavorare a temperature
criogeniche, questa tesi triennale si pone come scopo quello di introdurre ai concetti fondamentali della progettazione nel campo delle microonde necessari per
la progettazione e la successiva caratterizzazione di un semplice amplificatore RF, a titolo esplorativo verranno valutate anche le caratteristiche a bassa
temperatura.

\subsection{Transistor a eterogiunzione SiGe}
\subsection{Analisi dei circuiti a microonde e linee di trasmissione}
\section{Amplificatori RF e progettazione}

\subsection{Punto di Lavoro}
\label{sub_q_point}
La realizzazione di amplificatori operanti nel range delle microonde parte da una livello comune a tutti gli amplificatori: la scelta del punto di lavoro. Per poter operare correttamente e fornire un guadagno adeguato occorre polarizzare il dispositivo attivo che alimenta il nostro amplificatore, vanno quindi definite le correnti assorbite e le tensioni ai capi del transistor in condizioni statiche.

La polarizzazione dei transistor SiGe può essere trattata analogamente a quella dei normali BJT, considerando una configurazione ad emettitore comune in cui il terminale di connettore i due parametri principali che caratterizzano la risposta di piccolo segnale e che individuano il punto di lavoro sono la corrente assorbita dal collettore $I_C$ e la tensione fra collettore ed emettitore $V_{CE}$. Queste due variabili influenzano le prestazioni in termini di rumore, risposta in frequenza del transistor, guadagno, stabilità e dissipazione di potenza, ai fini di questa tesi solo le ultime due caratteristiche hanno condizionato la scelta della polarizzazione.

Definiti da progetto $I_{C}=\SI{8}{\milli\ampere}$, $V_{CE}=\SI{3.0}{\volt}$ e la tensione di alimentazione $V_{CC}=\SI{6}{\volt}$ e ricavati da datasheet la tensione base-collettore $V_{BE} = \SI{0.8}{\volt}$ e il guadagno statico di corrente $h_{FE} = \dfrac{I_C}{I_B}=160$ è stato possibile determinare la rete di polarizzazione che riportiamo di seguito:

!!!RETE DI POLSRIZZAZIONE!!!

\paragraph{Analisi della rete di polarizzazione}
La rete di polarizzazione è composta di una resistenza di base $R_{B1}$ e una di collettore $R_C$ che regolano rispettivamente tensione e corrente di base e di collettore, è presente anche una terza resistenza $R_{B2}$ che permette di variare la corrente di base indipendentemente dalla tensione di collettore. I due induttori $L_C$ e $L_B$ servono a fornire un percorso a bassa resistenza a corrente continua e ad alta impedenza alla frequenza di esercizio disaccoppiando il circuito di polarizzazione da quello di segnale.

I parametri del transistor come $V_{BE}$ e $h_{FE}$ dipendono fortemente dalla temperatura, per stabilizzare il punto di lavoro rispetto a possibili variazioni la rete di polarizzazione include un meccanismo di retroazione positiva. Se a causa di variazioni di $V_{BE}$ o $h_{FE}$ la corrente di collettore aumenta la caduta di tensione su $R_{C}$ aumenta e di conseguenza diminuisce la tensione $V_{BE}$, dall'equazione di Ebers-Molls si ricava che la corrente di collettore deve diminuire ristabilendo l'equilibrio.

!!!Eq di eb molls!!!

Per rendere questo meccanismo di retroazione effettivo occorre selezionare una resistenza $R_{C}>>h_{FE} \cdot R_{B1}$. Il calcolo delle resistenze è relativemte semplice e lo lasciamo di seguito.

!!! RICAVARE RESISTENZE !!!

I valori delle induttanze $L_C$ e $L_B$ sono stati scelti in modo da presentare una frequenza di autorisonanza di $\SI{6}{\giga\hertz}$ appena al di sopra della frequenza di lavoro di $\SI{5}{\giga\hertz}$. Gli induttori reali hanno sempre una capacità parassita parallela (EPC, \it{Equivalent Parallel Capacitance}) dovuta ad esempio alla capacità tra spire di avvolgimenti adiacenti, la frequenza alla quale la reattanza induttiva eguaglia in modulo la reattanza capacitiva dovuta ai parassiti viene chiamata frequenza di autorisonanza. A questa frequenza l'impedenza dell'induttore esplode disaccoppiando in maniera efficiente lo stadio di polarizzazione da quello di segnale.
Alla frequenza di risonanza:
\begin{equation}
    \vec{X_C}=\dfrac{-j}{2\pi f C_{EPC}}=-\vec{X_L}=2 \pi j f L
    \left|\vec{X_{TOT}}\right|=\left|\left(\dfrac{1}{\vec{X_C}}+\dfrac{1}{\vec{X_L}}\right)^{-1}\right|\to\infty
\end{equation} 

\subsection{Stabilità}



\section{Conclusione}







\end{document}