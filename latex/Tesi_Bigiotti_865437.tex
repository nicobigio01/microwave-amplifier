\documentclass[12pt]{article}
\usepackage[italian]{babel}
\usepackage{amsmath}
\usepackage{siunitx}
\usepackage{derivative}
\usepackage{graphicx}
\usepackage[a4paper, total={6in, 9in}]{geometry}
\usepackage{caption}
\usepackage{subcaption}
\usepackage{float}
\usepackage{derivative}

\begin{document}
\section{Introduzione}
\subsection{Motivazione e Contesto}
Negli ultimi decenni si è visto un notevole sviluppo della tecnologia legata ai sistemi a microonde dovuta a crescenti necessità sia industriali che di ricerca.
Quantum computing, radioastronomia e comunicazioni satellitari sono alcuni dei campi più importanti dove dispositivi a microonde devono lavorare a temperature
criogeniche, questa tesi triennale si pone come scopo quello di introdurre ai concetti fondamentali della progettazione nel campo delle microonde necessari per
la progettazione e la successiva caratterizzazione di un semplice amplificatore RF, a titolo esplorativo verranno valutate anche le caratteristiche a bassa
temperatura.

\subsection{Transistor a eterogiunzione SiGe}
\subsection{Analisi dei circuiti a microonde e linee di trasmissione}
\section{Progettazione e realizzazione dello schema}
\subsection{Struttura di un amplificatore RF}
La struttura degli amplificatori RF è abbastanza staandard, come per tutti gli amplificatori ci sono fasi di progetto bene distinte ma collegate: lo studio del punto di lavoro e quello di piccolo segnale.

Lo studio del punto di lavoro viene eseguito considerando solo le sorgenti a corrente continua spegnendo tutti i generatori di segnale, permettendo di studiare la polarizzazione dei dispositivi attivi. Lo studio di piccolo segnale ha come obiettivo lo studio dell'amplificatore in una regione linearizzata intorno al punto di lavoro, permette di definire le prestazioni dinamiche dell'amplificatore dopo aver spento tutti i generatori di polarizzazione.

Se consideriamo la parte di segnale possiamo ricavare una struttura a blocchi tipica per tutti gli amplificatori RF che riportiamo in figura,

!!! AMP RF !!!

sono ben evidenziati i due blocchi di adattamento (\textit{Matching Networks}) così come il transistor al centro. ciascuno di questi blocchi può essere visto come un \textit{two port network} caratterizzato da i rispettivi parametri di scattering e da due coefficienti di riflessione, uno per ogni verso di propagazione delle onde incidenti.

Per la progettazione dell'amplificatore torneranno utili i parametri di scattering del transistor, che vengono forniti dal produttore e contribuiscono a determinare i coefficienti di riflessione alle porte del transistor $\Gamma_{in}$ e $\Gamma_{out}$, tutto lo sforzo di progettazione sarà concentrato sulla progettazione delle reti di adattamento che determineranno i valori di $\Gamma_{S}$ e $\Gamma_{L}$.

Nel nostro caso le impedenze della sorgente di segnale e del carico sono tutte calibrate sull'impedenza caratteristica di $Z_{0}\SI{50}{\ohm}$.

\subsection{Punto di Lavoro}
\label{sub_q_point}
La realizzazione di amplificatori operanti nel range delle microonde parte da una livello comune a tutti gli amplificatori: la scelta del punto di lavoro. Per poter operare correttamente e fornire un guadagno adeguato occorre polarizzare il dispositivo attivo che alimenta il nostro amplificatore, vanno quindi definite le correnti assorbite e le tensioni ai capi del transistor in condizioni statiche.

La polarizzazione dei transistor SiGe può essere trattata analogamente a quella dei normali BJT, considerando una configurazione ad emettitore comune in cui il terminale di connettore i due parametri principali che caratterizzano la risposta di piccolo segnale e che individuano il punto di lavoro sono la corrente assorbita dal collettore $I_C$ e la tensione fra collettore ed emettitore $V_{CE}$. Queste due variabili influenzano le prestazioni in termini di rumore, risposta in frequenza del transistor, guadagno, stabilità e dissipazione di potenza, ai fini di questa tesi solo le ultime due caratteristiche hanno condizionato la scelta della polarizzazione.

Definiti da progetto $I_{C}=\SI{8}{\milli\ampere}$, $V_{CE}=\SI{3.0}{\volt}$ e la tensione di alimentazione $V_{CC}=\SI{6}{\volt}$ e ricavati da datasheet la tensione base-collettore $V_{BE} = \SI{0.8}{\volt}$ e il guadagno statico di corrente $h_{FE} = \dfrac{I_C}{I_B}=160$ è stato possibile determinare la rete di polarizzazione che riportiamo di seguito:

!!!RETE DI POLSRIZZAZIONE!!!

\paragraph{Analisi della rete di polarizzazione}
La rete di polarizzazione è composta di una resistenza di base $R_{B1}$ e una di collettore $R_C$ che regolano rispettivamente tensione e corrente di base e di collettore, è presente anche una terza resistenza $R_{B2}$ che permette di variare la corrente di base indipendentemente dalla tensione di collettore. I due induttori $L_C$ e $L_B$ servono a fornire un percorso a bassa resistenza a corrente continua e ad alta impedenza alla frequenza di esercizio disaccoppiando il circuito di polarizzazione da quello di segnale.

I parametri del transistor come $V_{BE}$ e $h_{FE}$ dipendono fortemente dalla temperatura, per stabilizzare il punto di lavoro rispetto a possibili variazioni la rete di polarizzazione include un meccanismo di retroazione positiva. Se a causa di variazioni di $V_{BE}$ o $h_{FE}$ la corrente di collettore aumenta la caduta di tensione su $R_{C}$ aumenta e di conseguenza diminuisce la tensione $V_{BE}$, dall'equazione di Ebers-Molls si ricava che la corrente di collettore deve diminuire ristabilendo l'equilibrio.

\begin{equation}
    I_C=I_S(T)\left(e^{V_{BE}/V_T}-1\right)
\end{equation}

Per rendere questo meccanismo di retroazione effettivo occorre selezionare una resistenza $R_{C}>>h_{FE} \cdot R_{B1}$. Il calcolo delle resistenze è relativemte semplice e lo lasciamo di seguito.

!!! RICAVARE RESISTENZE !!!

I valori delle induttanze $L_C$ e $L_B$ sono stati scelti in modo da presentare una frequenza di autorisonanza di $\SI{6}{\giga\hertz}$ appena al di sopra della frequenza di lavoro di $\SI{5}{\giga\hertz}$. Gli induttori reali hanno sempre una capacità parassita parallela (EPC, \textit{Equivalent Parallel Capacitance}) dovuta ad esempio alla capacità tra spire di avvolgimenti adiacenti, la frequenza alla quale la reattanza induttiva eguaglia in modulo la reattanza capacitiva dovuta ai parassiti viene chiamata frequenza di autorisonanza. A questa frequenza l'impedenza dell'induttore esplode disaccoppiando in maniera efficiente lo stadio di polarizzazione da quello di segnale.
Alla frequenza di risonanza:
\begin{equation}
    \begin{split}
        \vec{X_C}=\dfrac{-j}{2\pi f C_{EPC}}=-\vec{X_L}=2 \pi j f L\\
        \left|\vec{X_{TOT}}\right|=\left|\left(\dfrac{1}{\vec{X_C}}+\dfrac{1}{\vec{X_L}}\right)^{-1}\right|\to\infty 
    \end{split}
\end{equation} 

\subsection{Stabilità}
La progettazione di un amplificatore deve tenere conto delle potenziali instabilità che posso sorgere a causa dei motivi più disparati. In presenza di una instabilità un segnale in ingresso può viene amplificato senza limiti, fin quando l'amplificatore non satura o si danneggia o inizia a oscillare.

Negli amplificatori RF lo studio della stabilità risulta insidioso. Ad alta frequenza esiste il fenomeno delle riflessioni, non tutta la potenza incidente su una porta viene trasmessa, una parte può essere riflessa. Quando si considerano dispositivi attivi è possibile imbattersi in una situazione per cui si ha \textit{reflected gain}, ovvero il segnale riflesso risulta amplificato rispetto a quello incidente.

Se consideriamo la (!!!REF!!!) la presenza di \textit{reflected gain} è data da
\begin{equation}
    \left|\Gamma_{in}\right| > 1
\end{equation}
\begin{equation}
    \left|\Gamma_{out}\right| > 1
\end{equation}
rispettivamente per le porte di ingresso e uscita del transistor.

Le reti di adattamento vengono solitamente realizzate con elementi passivi, quindi vale
\begin{equation}
    \left|\Gamma_{S}\right| < 1
\end{equation}
\begin{equation}
    \left|\Gamma_{L}\right| < 1
\end{equation}

Per avere una condizione di instabilità devono quindi le seguenti condizioni che impongono la presenza di un guadagno ad anello nella sezione di ingresso o uscita, la condizione sulla fase serve a garantire che avvenga interferenza costruttiva tra i segnali incidenti e riflessi.
\begin{equation}
    \left|\Gamma_{in}\Gamma_{S}\right| > 1
\end{equation}
\begin{equation}
    \angle\Gamma_{in}\Gamma_{S} = \SI{0}{\degree}
\end{equation}
\begin{equation}
    \left|\Gamma_{out}\Gamma_{L}\right| > 1
\end{equation}
\begin{equation}
    \angle\Gamma_{out}\Gamma_{L} = \SI{0}{\degree}
\end{equation}

Studiando il percorso del segnale tra i vari blocchi dell'amplificatore di (REF alla figura) e con un po' di algebra si può ricavare i valori di $\left|\Gamma_{in}\right|$ e $\left|\Gamma_{out}\right|$
\begin{equation}
    \left|\Gamma_{in}\right| = \left|S_{11} + \dfrac{S_{12}S_{21}\Gamma_{L}}{1-S_{22}\Gamma_{L}}\right|
\end{equation}
\begin{equation}
    \left|\Gamma_{out}\right| = \left|S_{22} + \dfrac{S_{12}S_{21}\Gamma_{S}}{1-S_{11}\Gamma_{S}}\right|
\end{equation}
La cosa interessante che si nota è che i coefficienti di riflessione non dipendono soltanto dai parametri del transistor ma anche dalle reti di adattamento di impedenza, questo perché il segnale incidente sull'ingresso del transistor può essere trasmesso fino all'uscita dove apparirà amplificato, raggiunta la rete di adattamento di uscita solo una parte raggiungerà il carico mentre la rimanente risulterà riflessa e percorrerà la strada opposta fino a presentarsi nuovamente all'ingresso del transistor sommandosi alla potenza riflessa in ingresso e contribuendo a determinare il coefficiente di riflessione. Ragionamento analogo vale per $\Gamma_{out}$. Chairamente tutto questo vale solo se il transistor presenta $\left|S_{12}\right|\neq 0$.

La combinazione dei parametri di scattering per una particolare frequenza e punto di lavoro può essere tale da garantire $\left|\Gamma_{in}\right| < 1$ e $\left|\Gamma_{out}\right| < 1$ indipendentemente dalla scelta di $\Gamma_{L}$ e $\Gamma_{S}$, se questa condizione è verificata si parla di stabilità incodizionata del transistor. Esistono vari test per verificare la stabilità incodizionata noti i parametri di scattering, uno di questi consiste nel verificare che la grandezza
\begin{equation}
    \mu = \dfrac{1-\left|S_{22}\right|^2|}{\left|S_{11}-S_{22}^{*}\Delta\right|+\left|S_{21}S_{12}\right|}> 1 \hspace{10pt} \text{con} \hspace{10pt} \Delta=S_{11}S_{22}-S_{21}S_{12}
\end{equation}

Nel nostro caso è stato possibile valutare $\mu$ lungo tutto lo spettro di frequenze per il transisor BFP640 polarizzato a $I_{C}=\SI{8}{\milli\ampere}$, $V_{CE}=\SI{3.0}{\volt}$ nel range di frequenze da $\SI{30}{\mega\hertz}$ a $\SI{10}{\giga\hertz}$ utilizzando MATLAB, riportiamo il risultato:
!!!GRAFICO!!!
Come si può notare per il range di frequenze di nostro interesse fra $\SI{4.5}{\giga\hertz}$ a $\SI{5.5}{\giga\hertz}$ il transistor risulta essere stabile. Si può anche notare il tipico andamento della stabilità del transistor, a basse frequenze il transistor può essere considerato unilatero e la stabilità migliora notevolmente, ad alte frequenze si ha un uguale miglioramento dovuto ad un valore di $S_{21}$ sempre più piccolo.

!!! SECONDO AMP !!!

\subsection{Reti di adattamento e Studio del guadagno}
Negli amplificatori RF il guadagno totale non dipende solo dal transistor ma anche dalle reti di adattamento che permettono di massimizzare il trasferimento di potenza dalla sorgente al transistor e dal transistor al carico.
Vengono definiti tre guadagni, $G_P$ rappresenta il rapporto fra la potenza ai capi del transistor e la potenza dissipata sul carico,
$G_A$ è il rapporto fra la potenza fornita dalla sorgente in ingresso e quella disponibile all'uscita del transistor, infine $G_T$ è il rapporto tra la potenza fornita dalla sorgente e quella che viene dissipata sul carico.
\begin{equation}
    G_A = \dfrac{\left|S_{21}\right|^2\left(1-\left|\Gamma_{S}\right|^2\right)}{\left|1-S_{11}\Gamma_{S}\right|^2\left(1-\left|\Gamma_{out}^2\right|\right)}
\end{equation}
\begin{equation}
    G_P = \dfrac{\left|S_{21}\right|^2\left(1-\left|\Gamma_{L}\right|^2\right)}{\left|1-S_{11}\Gamma_{L}\right|^2\left(1-\left|\Gamma_{in}^2\right|\right)}
\end{equation}
\begin{equation}
    G_0 = \left|S_{21}\right|^2
\end{equation}
Queste ulteriori definizioni di guadagno sono utili in quanto non dipendono dall'adattamento di ingresso o di uscita nel caso rispettivamento di $G_P$ e $G_A$. Chiaramente il guadagno totale dell'amplificatore $G_T$ è influenzato sia dalle sezioni d'ingresso e d'uscita dell'amplificatore.

Il transistor ha un suo guadagno intrinseco definito da $S_{21}$, per definizione dei parametri di scattering questo guadagno è ottenuto terminando ingresso e uscita all'impedenza caratteristica di $\SI{50}{\ohm}$. In queste condizioni l'amplificatore è parecchio limitato in termini di guadagno in quanto ad alte frequenze le impedenze di ingresso ed uscita del transistor terminate a $\SI{50}{\ohm}$ riflettono buona parte del segnale da amplificare.

Per risolvere il problema vengono introdotte le reti di adattamento di impedenza che permettono di adattare l'impedenza caratteristica della linea a quelle del transistor. Per ottenere il massimo guadagno viene eseguito un adattamento coniugato, ponendo il coefficiente di riflessione $\Gamma_{L}$ e $\Gamma_{S}$ pari al complesso coniugato rispettivamente di $\Gamma_{in}$ e $\Gamma_{out}$. L'espressioni di $\Gamma_{L}$ e $\Gamma_{S}$ per ottenere adattamento coniugato si ricavano con qualche passaggio algebrico,
!!! GAMMA IN E...!!!

Nel all'amplificatore con stub l'adattamento di impedenza è stato realizzato con la tecnica a singolo stub. Essendo incodizionatamente stabile è stato possibile perseguire una progettazione per massimo guadagno.

Con MATLAB sono stati ricavati i valori di $\Gamma_{S}$ e $\Gamma_{L}$ noti i valori dei parametri di scattering, una volta rappresentati sulla carta di Smith si può procedere al dimensionamento delle linee di trasmissione. 

!!!DA CONCLUDERE!!!

Il procedimento è leggermente diverso per il secondo amplificatore, avendo definito nella sezione precedente le zone stabili si può procedere con lo studio del guadagno.
In questo caso vogliamo realizzare un amplificatore che abbia un guadagno di $G_T=\SI{22}{\decibel}$, esistono varie combinazioni di $G_P$ e $G_A$ che permettono di ottenere lo stesso risultato, in questo caso abbiamo scelto $G_P=\SI{25}{\decibel}$ e $G_A=\SI{22}{\decibel}$, sono stati quindi disegnate le due circoferenze che individuano tutti i coefficienti di riflessione $\Gamma_{S}$ e $\Gamma_{L}$ che permettono di ottenere rispettivamente i valori $G_P$ e $G_A$ voluti.

!!!IMMAGINE MATLAB!!!

Il valore esatto dei due coefficienti di riflessione andrebbe scelto con riguardo a considerazioni di rumore che però andavano oltre lo scopo di questo elaborato, la scelta è stata quindi dettata da ragioni di tipo pratico, riportiamo i due diagrammi di smith relativi rispettivamente alla rete di ingresso e di uscita.

!!! MATCHING !!!

Il risultato simultato dei parametri di scattering è stato ottenuto con MATLAB e lo riportiamo di seguito

!!! GRAFICO !!!

Si può notare il guadagno pari a $\approx\SI{22}{\decibel}$ alla frequenza di progetto, non trattandosi di un amplificatore progettato per il massimo guadagno $S_{11}$ e $S_{22}$ presentano un modulo leggermente più grande rispetto al caso precedente.

% Riprendendo la figura (!!!REF!!!) si possono assegnare tre guadagni così definiti
% \begin{equation}
%     G_S = \dfrac{1-\left|\Gamma_{S}\right|^2}{\left|1-S_{11}\Gamma_{S}\right|^2}
% \end{equation}
% \begin{equation}
%     G_L = \dfrac{1-\left|\Gamma_{L}\right|^2}{\left|1-S_{22}\Gamma_{L}\right|^2}
% \end{equation}
% \begin{equation}
%     G_0 = \left|S_{21}\right|^2
% \end{equation}
% \subsection{Schema finale}

\section{Progettazione della scheda stampata}
\section{Analisi dei risultati}


\section{Conclusione}







\end{document}